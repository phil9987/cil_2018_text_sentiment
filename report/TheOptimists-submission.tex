\documentclass[10pt,conference,compsocconf]{IEEEtran}

%\usepackage{times}
%\usepackage{balance}
\usepackage{url}
\usepackage{graphicx}
\usepackage{color}

\newcommand{\todo}[1]{}
\renewcommand{\todo}[1]{{\color{red} TODO: {#1}}}

\begin{document}
\title{CIL 2018: Text Sentiment Classification}

\author{
  Pirmin Schmid, Philip Junker, Nikolas Göbel, Aryaman Fasciati\\
  The Optimists\\
  Department of Computer Science, ETH Zurich, Switzerland
}

\maketitle

\begin{abstract}
  Lorem ipsum dolor sit amet, consectetur adipiscing elit, sed do
  eiusmod tempor incididunt ut labore et dolore magna aliqua. Ut enim ad
  minim veniam, quis nostrud exercitation ullamco laboris nisi ut
  aliquip ex ea commodo consequat. Duis aute irure dolor in
  reprehenderit in voluptate velit esse cillum dolore eu fugiat nulla
  pariatur. Excepteur sint occaecat cupidatat non proident, sunt in
  culpa qui officia deserunt mollit anim id est laborum.
\end{abstract}


\section{Introduction}

The goal of this project is to build a sentiment classifier
that predicts whether a tweet text used to include a
positive smiley :) or a negative smiley :(,
based on the remaining text.

Our first baseline uses random forests and achieved an accuracy of 72\%.
Our second baseline uses a Recurrent Neural Network (RNN) model with
an accuracy of TODO\%.

In a third model, we refined our second baseline, incorporating
\todo{describe novel approach}, in a RNN-based approach.
This model achieved \todo{\%} accuracy.


\section{Related Work}
Write about related work \cite{go2016mastering}


\section{Models}

\subsection{First Baseline (B1)}

Our first baseline uses a random forest model with unlimited max\_depth and 20 estimators.
Each tweet is represented by the the average of its word embedding vectors.
We used pretrained GloVe \cite{glove} embeddings from Stanford,
together with a slightly adapted version of the preprocessor script
provided by Stanford.
Words that are not in the vocabulary are ignored.

This model achieved an accuracy of 72\%.


\subsection{Second Baseline (B2)}

\section*{Acknowledgements}
The authors wish to express their gratitude to the Euler and Leonhard
clusters at ETH, whose unwavering computational effort this project
could not have done without.

\bibliographystyle{IEEEtran}
\bibliography{TheOptimists-literature}
\end{document}
